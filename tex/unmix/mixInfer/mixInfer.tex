\documentclass{article}
\usepackage{amsfonts}
\usepackage{amsmath}
\usepackage[cm]{fullpage}
\usepackage{graphicx}
\usepackage{hyperref}
\usepackage{xcolor}
\graphicspath{
{/home/jburdick/gcb/work/svnroot/trunk/src/}
{/home/jburdick/gcb/}
}
\begin{document}


\section{Estimating the sort matrix}

In previous simulations, we assumed that the sort matrix was
known exactly. However, in practice, we don't necessarily know
exactly which cells were included in a given sorted fraction.
For instance, we don't know exactly what stages of embryos are
sorted, which cells are lost because of sorting, or how reporter
intensities differ between microscopy and FACS sorting.


\subsection{Notation}

First, we define some variables:

\begin{itemize}

\item Expression: $X_{ij}$ is the expression of gene $i$ in cell $j$

\item Sorting: $S_{fj}$ is the proportion of cell $j$ in sorted fraction $f$

\item Volume: $V_j$ is the volume of cell $j$

\end{itemize}

Here, $X$, $S$, and $V$ are all relative to the entire embryo sample.
This is a change in units (previously, we were measuring $X$ as
the proportion of a gene's expression which was in a given cell.)

Note that all of these variables are constrained to be positive.

\subsection{Constraints}

All the variables listed above must be positive.
We now list constraints needed to phrase the problem as a
constrained linear system (although potentially {\em much} larger than
before; we ignore this efficiency issue for now.)

\subsubsection{The expression data}

We know that the total expression of all the genes in
a cell add up to that cell's volume:

\[
\sum_i X_{ij} = V_j
\]

\color{red}

INCORRECT STUFF

Suppose we have measured that the expression of a gene $i$ in sorted
fraction $f$ is $a$ (measured as a proportion; so 1 ppm would correspond
to $a = 10^{-6}$.) Then

\[
\sum_j X_{ij} = a S_{fi}
\]

Note that this constraint assumes that the expression data
is known exactly. This is incorrect; there is certainly
uncertainty in its measurement. However, noise in RNA-seq experiments
may be known more precisely than which cells are going to be
included in cell sorting, or the relative volumes of
different cells.

\color{black}

\subsubsection{The sort matrix}

We could encode the fact that a given sorted fraction $f$ always,
or never,
includes a cell $j$ by adding constraints

\[
S_{fj} = V_j \hspace{5pt} {\textrm or} \hspace{5pt} S_{fj} = 0
\]

respectively. However, estimating these entries is a principal point
of this model. We could add a noisy constraint that
$S_{fj} \sim \mathcal{N}(0, \sigma)$, but this wouldn't scale with $V_j$.


One way to encode this is by using a prior which is no longer diagonal,
but includes covariance. If we write, for example

\[
\begin{bmatrix}
V_j
S_{fj}
\end{bmatrix}
\sim \mathcal{N}(
\begin{bmatrix}
0 \\
0
\end{bmatrix}
,
\begin{bmatrix}
1 & -0.5 \\
-0.5 & 1 
\end{bmatrix}
)
\]

then $S_{fj}$ will be ``pulled toward'' $V_j$, but not actually
constrained to be equal to it.

(We could probably use a similar method to encode noise in the
read measurements.)

\subsubsection{Constraints on volume}

In the (fairly common) case in which we have matched positive and negative
sorted samples, which we'll call $f^+$ and $f^-$, we have

\[
S_{f^+j} + S_{f^-j} = V_j
\]

The fact that this is exact seems like a relatively safe assumption:
in a given FACS experiment, a particular cell presumably ended up
in one tube or the other.
This generalizes to double-sorted fractions.

Since both $S_{f^+j}$ and $S_{f^-j}$ are constrained to be positive,
this implies that they are both less than $V_j$. If we have sort fractions
without a negative control, then presumably we need to add a constraint

\[
S_{fj} \le V_j
\]

Lastly, the volumes of the cells all should add up to 1:

\[
\sum_j V_j = 1
\]



\section{Simulating / testing this framework}


Questions we address:

How much does incorrect assumptions about cell volume affect accuracy?
(``Missing cells'' is considered a special case with volume zero.)

Can we infer correct cell volumes / missing cells?



\subsection{The example organism}

We wish to test these ideas using the existing EP solver. However, it
seems that this phrasing of the problem requires solving for expression
in all cells at once. Therefore, we first test using a scaled-down example;
if this works, then we can see if the EP solver can be modified to handle
the actual data.

The fictional organism has 31 cells, and develops in an invariant lineage
(a binary tree.) We assume that we have measured expression in






\section{Other stuff}

This may not be relevant.

\subsection{Connection to non-negative matrix factorization}

Let $A$ be the sort matrix, $X$ be the matrix of per-cell expression, and
$B$ be the matrix of expression data. We are essentially trying to factor
$AX=B$, with $A, X$, and $B$ non-negative.

This is reminiscent of non-negative matrix factorization, but
there are important differences. First, in our situation,
the dimensions of $X$ are larger than $B$. Also, we have some
prior information about
the matrix $A$, from the microscopy data.
We'd like to use this to constrain our estimate of $A$.




\subsection{Using a Dirichlet prior}

Erkkila {\em et al} use a Dirichlet prior on $A$, which seems reasonable. Trying to do
this, we add a term $t_M(x)$, which encodes the Dirichlet prior $\alpha$.

\footnote{The fact that so many things are proportions makes it tempting
to use something like Dirichlet distributions; however, the variances
for several of these may be different -- we may know that some cells
are definitely in a fraction, but others we may think are possibly 
in the fraction, and possibly not.}

\begin{eqnarray}
t_M(A) & = & \mathrm{Dirichlet}(A|\alpha) \\
t_A(x) & = & [Ax = b] \\
t_i(x) & = & [x_i \ge 0]
\end{eqnarray}

Now, the product representing the (again, improper) probability distribution
whose marginals we are estimating:

\begin{eqnarray}
p(A, x|\alpha) & \propto & t_M(A) t_A(x) \prod_i t_i(x)
\end{eqnarray}

Sampling from this seems painful, as it seems to require sampling all of
the $x_i$, and $A$, simultaneously.





\end{document}

