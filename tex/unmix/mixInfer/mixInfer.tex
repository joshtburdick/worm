\documentclass{article}
\usepackage{amsfonts}
\usepackage{amsmath}
\usepackage[cm]{fullpage}
\usepackage{graphicx}
\usepackage{hyperref}
\graphicspath{
{/home/jburdick/gcb/work/svnroot/trunk/src/}
{/home/jburdick/gcb/}
}
\begin{document}


\section{Estimating the sort matrix}


Another way to phrase the problem is to let $X_{ij}$ represent
a {\em proportion of the expression in a given cell}, rather than
a proportion of the total expression in the embryo.

Suppose that

$S_{si}$ is the fraction (relative to the whole embryo)
of cell $i$ in fraction $s$,

$X_{ij}$ is the fraction of gene $j$ in cell $i$, and

$B_{sj}$ is the total expression of gene $j$ in fraction $s$.

Here, $S$ and $X$ are unknown; we model them with normal
distributions.\footnote{The fact that so many things are proportions makes it tempting
to use something like Dirichlet distributions; however, the variances
for several of these may be different (we may know that some cells
are definitely in a fraction, but others we may think are possibly 
in the fraction, and possibly not.}

$B$, on the other hand, is known. There is some uncertainty in its
measurement, but noise in RNA-seq experiments is arguably better understood
than which cells are going to included in cell sorting, or the relative
sizes of different cells.

For now, we assume that $B$ is known {\em precisely}. Thus, 


\subsection{Modelling cell volume}

Suppose that for each cell $i$ we have an estimate $V_i$ of the mean and
variance of its volume. If we have a variable for each cell's volume, then
we can incorporate this as a ``soft'' constraint. We also have the constraint
that $\sum_i V_i = 1$.

We can sort of encode the assumption that ``all of a cell $i$ is included
in a sort fraction $s$'' by adding a constraint that $S_{si} = V_i$
(or, trivially, ``none of a cell $i$ is included in a sort fraction $s$''
as $S_{si} = 0$.) These are ``hard'' constraints, for better or worse;
we could add some variance, but if it's constant, it won't scale with
the volume.

Presumably, predictions that ``cells are missing'' should correspond to
solutions with $V_i \approx 0$.





\section{Other stuff}

This may not be relevant.

\subsection{Including mixture proportions}

In practice, we may not know the ``sort matrix'' $A$ precisely.
Let $A$ be the sort matrix, $X$ be the matrix of per-cell expression, and
$B$ be the matrix of expression data. We are essentially trying to factor
$AX=B$, with $A, X$, and $B$ non-negative.

This is reminiscent of non-negative matrix factorization, but
there are important differences. First, in our situation,
the dimensions of $X$ are larger than $B$. Also, we have some
prior information about
the matrix $A$, from the microscopy data.
We'd like to use this to constrain our estimate of $A$.

Side note: covariance or correlation is somewhat related to
the product of variables.

\subsection{The EP problem, using Gaussians}

We are trying to estimate marginals of the distribution of $x$, given the
constraints that $Ax = b$ and $x \ge 0$. We can express these constraints
as ``indicator functions'':

\begin{eqnarray}
t_A(x) & = & [Ax = b] \\
t_i(x) & = & [x_i \ge 0]
\end{eqnarray}

Their product is the (improper) probability distribution whose marginals
we are estimating:

\begin{eqnarray}
p(x) & \propto & t_A(x) \prod_i t_i(x)
\end{eqnarray}

EP is applicable when we're approximating a product of terms. We approximate each of the terms with
Gaussians
$\tilde{t}_i(x)$ and $\tilde{t}_A(x)$ (although none of these
terms, individually, is at all Gaussian.) 

\begin{eqnarray}
q(x) & \propto & \tilde{t}_A(x) \prod_i \tilde{t}_i(x)
\end{eqnarray}

As in Cseke and Heskes, rather than updating each term individually,
we update all of the $\tilde{t}_i(x)$ terms in parallel, and then factor in
the linear constraint $\tilde{t}_A(x)$.


\subsection{Using a Dirichlet prior}

Erkkila {\em et al} use a Dirichlet prior on $A$, which seems reasonable. Trying to do
this, we add a term $t_M(x)$, which encodes the Dirichlet prior $\alpha$.

\begin{eqnarray}
t_M(A) & = & \mathrm{Dirichlet}(A|\alpha) \\
t_A(x) & = & [Ax = b] \\
t_i(x) & = & [x_i \ge 0]
\end{eqnarray}

Now, the product representing the (again, improper) probability distribution
whose marginals we are estimating:

\begin{eqnarray}
p(A, x|\alpha) & \propto & t_M(A) t_A(x) \prod_i t_i(x)
\end{eqnarray}

Sampling from this seems painful, as it seems to require sampling all of
the $x_i$, and $A$, simultaneously.


\end{document}

