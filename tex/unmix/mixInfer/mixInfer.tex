\documentclass{article}
\usepackage{amsfonts}
\usepackage{amsmath}
\usepackage[cm]{fullpage}
\usepackage{graphicx}
\usepackage{hyperref}
\graphicspath{
{/home/jburdick/gcb/work/svnroot/trunk/src/}
{/home/jburdick/gcb/}
}
\begin{document}


\section{Estimating the sort matrix}

In previous simulations, we assumed that the sort matrix was
known exactly. However, in practice, we don't necessarily know
exactly which cells were included in a given sorted fraction.
For instance, we don't know exactly what stages of embryos are
sorted, which cells are lost because of sorting, or how reporter
intensities differ between microscopy and FACS sorting.

\subsection{Notation}

Suppose that

$V_i$ is the volume of cell $i$

$S_{si}$ is the proportion of cell $i$ in sorted fraction $s$

$X_{ij}$ is the proportion of gene $j$ in cell $i$

$B_{sj}$ is the total expression of gene $j$ in sorted fraction $s$

\vspace{4mm}

Note that $V$, $S$, and $X$ are all relative to the entire embryo sample.
This is a change in units (previously, we were measuring $X$ as
the proportion of a gene's expression which was in a given cell.)

$V$, $S$ and $X$ are unknown; we model them with normal
distributions.\footnote{The fact that so many things are proportions makes it tempting
to use something like Dirichlet distributions; however, the variances
for several of these may be different -- we may know that some cells
are definitely in a fraction, but others we may think are possibly 
in the fraction, and possibly not.}

$B$, on the other hand, we consider mostly known. There is some
uncertainty in its measurement, but noise in RNA-seq experiments
is arguably better understood than which cells are going to be
included in cell sorting, or the relative sizes of different cells.

\subsection{Constraints}

We now list constraints needed to phrase the problem in the same
constrained linear form as before (although potentially as a larger
problem.) All the variables listed before must be positive.

\subsection{Modelling cell volume}

In the above, we haven't explicitly included the effect of cell volume.
However, several of the variables in the above are related: we may know
that a given cell expresses several sort markers strongly, but that cell
may not be included in the sorting process.

Presumably, predictions that ``cells are missing'' should correspond to
a posterior with $V_i \approx 0$.

Suppose that for each cell $i$ we have an estimate $V_i$ of the mean and
variance of its volume. If we have a variable for each cell's volume, then
we can incorporate this as a ``soft'' constraint. We also have the constraint
that $\sum_i V_i = 1$.

We can sort of encode the assumption that ``all of a cell $i$ is included
in a sort fraction $s$'' by adding a constraint that $S_{si} = V_i$
(or, trivially, ``none of a cell $i$ is included in a sort fraction $s$''
as $S_{si} = 0$.) These are ``hard'' constraints, for better or worse;
we could add some variance, but if it's constant, it won't scale with
the volume.




\subsection{Modelling which cells are sorted}







\section{Simulating / testing this framework}


Questions we address:

How much does incorrect assumptions about cell volume affect accuracy?
(``Missing cells'' is considered a special case with volume zero.)

Can we infer correct cell volumes / missing cells?



\subsection{The example organism}

We wish to test these ideas using the existing EP solver. However, it
seems that this phrasing of the problem requires solving for expression
in all cells at once. Therefore, we first test using a scaled-down example;
if this works, then we can see if the EP solver can be modified to handle
the actual data.

The fictional organism has 31 cells, and develops in an invariant lineage
(a binary tree.) We assume that we have measured expression in






\section{Other stuff}

This may not be relevant.

\subsection{Connection to non-negative matrix factorization}

Let $A$ be the sort matrix, $X$ be the matrix of per-cell expression, and
$B$ be the matrix of expression data. We are essentially trying to factor
$AX=B$, with $A, X$, and $B$ non-negative.

This is reminiscent of non-negative matrix factorization, but
there are important differences. First, in our situation,
the dimensions of $X$ are larger than $B$. Also, we have some
prior information about
the matrix $A$, from the microscopy data.
We'd like to use this to constrain our estimate of $A$.

\subsection{The EP problem, using Gaussians}

We are trying to estimate marginals of the distribution of $x$, given the
constraints that $Ax = b$ and $x \ge 0$. We can express these constraints
as ``indicator functions'':

\begin{eqnarray}
t_A(x) & = & [Ax = b] \\
t_i(x) & = & [x_i \ge 0]
\end{eqnarray}

Their product is the (improper) probability distribution whose marginals
we are estimating:

\begin{eqnarray}
p(x) & \propto & t_A(x) \prod_i t_i(x)
\end{eqnarray}

EP is applicable when we're approximating a product of terms. We approximate each of the terms with
Gaussians
$\tilde{t}_i(x)$ and $\tilde{t}_A(x)$ (although none of these
terms, individually, is at all Gaussian.) 

\begin{eqnarray}
q(x) & \propto & \tilde{t}_A(x) \prod_i \tilde{t}_i(x)
\end{eqnarray}

As in Cseke and Heskes, rather than updating each term individually,
we update all of the $\tilde{t}_i(x)$ terms in parallel, and then factor in
the linear constraint $\tilde{t}_A(x)$.


\subsection{Using a Dirichlet prior}

Erkkila {\em et al} use a Dirichlet prior on $A$, which seems reasonable. Trying to do
this, we add a term $t_M(x)$, which encodes the Dirichlet prior $\alpha$.

\begin{eqnarray}
t_M(A) & = & \mathrm{Dirichlet}(A|\alpha) \\
t_A(x) & = & [Ax = b] \\
t_i(x) & = & [x_i \ge 0]
\end{eqnarray}

Now, the product representing the (again, improper) probability distribution
whose marginals we are estimating:

\begin{eqnarray}
p(A, x|\alpha) & \propto & t_M(A) t_A(x) \prod_i t_i(x)
\end{eqnarray}

Sampling from this seems painful, as it seems to require sampling all of
the $x_i$, and $A$, simultaneously.


\subsection{Dirichlet, take 2}

Suppose we model each cell $x_i$ as a Dirichlet, and the mixture as a Dirichlet.
Also, we model the mixture for each fraction $j$ as a Dirichlet.

\begin{eqnarray}
x_i & \sim & \mathrm{Dirichlet}(\mathbf{1}) \\

\end{eqnarray}





\end{document}

