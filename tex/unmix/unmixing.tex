\documentclass{article}
\usepackage{amsfonts}
\usepackage[cm]{fullpage}
\usepackage{graphicx}
\usepackage{hyperref}
\graphicspath{
{/home/jburdick/gcb/work/svnroot/trunk/src/}
{/home/jburdick/gcb/}
}
\begin{document}

\subsection*{Cell sorting matrix}

The unmixing methods require an estimate of which cell is in which sorted fraction,
and how much RNA each cell contributes to the sample.

Let $S_{ij}$ be
the probability that a given cell was present in a given sorted fraction. Let $V_j$
be the volume of the $j$th cell. Then we define

\[
M_{ij} = \frac{S_{ij} V_j}{ \sum\limits_{i}^{} S_{ij} V_j }
\]


\subsubsection*{Image thresholding}

We estimated $S_{ij}$ based on 29 movies.
For each movie, we manually chose an intensity distribution for cells expressing
or not expressing the reporter, and 
used this logistic model to classify each cell as ``expressing'' or ``not expressing.''
We assumed that the probability of a cell being in a negative fraction was
one minus the probability of a cell being in a positive fraction.

\subsubsection*{Estimating cell volume}

We assumed that the amount of RNA in each cell was proportional to the cell's volume,
times the number of time steps that the cell existed.
We estimated cell volume by assuming that each division was exactly equal (although
this is clearly approximate), and so

\[
V_j = 2^{-\mathrm{branchlength}}
\]

\subsubsection*{Estimating biological noise between measurements}

We estimated biological noise between replicates using four samples which were
grown and sorted independently. For each (on log scale), we plotted the variance
in read depth between replicates, as a function of average read depth.

\includegraphics[width=0.7\textwidth]{git/unmix/seq/quant/noiseBetweenReplicates.pdf}

The average slope among these experiments (assuming a zero $y$-intercept) was 1.05.
Therefore (for the sake of simplicity)
we assumed that the variance equalled the number of reads.

\subsubsection*{Sorting purity}

The sorting purities (measured by re-sorting the sorted cells) ranged from 82\% to 97\%.
This means that the measured depletion of a gene in a given sorted fraction is less than
the actual depletion. (This is important, because we're using depletion of a gene in a
sorted fraction to ``rule out'' expression in some cells.)

Let $S$ be the measured expression of a gene in the singlet cells,
$F$ be the
measured expression of a gene in a sorted fraction, with known sort purity $p$, and
$G$ be the true expression of that gene in the sorted fraction.
We can model $F$ as a mixture of $S$ and $G$:

\[
F = pG + (1-p)S
\]

If we assume that $S$ and $F$ are normally distributed (with variance estimated as described
above), then we can obtain $G$'s distribution, which is also normal.

Note that this can result in estimates of $G$ which are negative; in such
cases, we set the mean to zero.

\subsection*{Unmixing using the pseudoinverse}





\subsection*{Unmixing using EP}






\end{document}

