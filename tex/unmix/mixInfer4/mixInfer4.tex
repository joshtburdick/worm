\documentclass{article}\usepackage[]{graphicx}\usepackage[]{color}
%% maxwidth is the original width if it is less than linewidth
%% otherwise use linewidth (to make sure the graphics do not exceed the margin)
\makeatletter
\def\maxwidth{ %
  \ifdim\Gin@nat@width>\linewidth
    \linewidth
  \else
    \Gin@nat@width
  \fi
}
\makeatother

\definecolor{fgcolor}{rgb}{0.345, 0.345, 0.345}
\newcommand{\hlnum}[1]{\textcolor[rgb]{0.686,0.059,0.569}{#1}}%
\newcommand{\hlstr}[1]{\textcolor[rgb]{0.192,0.494,0.8}{#1}}%
\newcommand{\hlcom}[1]{\textcolor[rgb]{0.678,0.584,0.686}{\textit{#1}}}%
\newcommand{\hlopt}[1]{\textcolor[rgb]{0,0,0}{#1}}%
\newcommand{\hlstd}[1]{\textcolor[rgb]{0.345,0.345,0.345}{#1}}%
\newcommand{\hlkwa}[1]{\textcolor[rgb]{0.161,0.373,0.58}{\textbf{#1}}}%
\newcommand{\hlkwb}[1]{\textcolor[rgb]{0.69,0.353,0.396}{#1}}%
\newcommand{\hlkwc}[1]{\textcolor[rgb]{0.333,0.667,0.333}{#1}}%
\newcommand{\hlkwd}[1]{\textcolor[rgb]{0.737,0.353,0.396}{\textbf{#1}}}%

\usepackage{framed}
\makeatletter
\newenvironment{kframe}{%
 \def\at@end@of@kframe{}%
 \ifinner\ifhmode%
  \def\at@end@of@kframe{\end{minipage}}%
  \begin{minipage}{\columnwidth}%
 \fi\fi%
 \def\FrameCommand##1{\hskip\@totalleftmargin \hskip-\fboxsep
 \colorbox{shadecolor}{##1}\hskip-\fboxsep
     % There is no \\@totalrightmargin, so:
     \hskip-\linewidth \hskip-\@totalleftmargin \hskip\columnwidth}%
 \MakeFramed {\advance\hsize-\width
   \@totalleftmargin\z@ \linewidth\hsize
   \@setminipage}}%
 {\par\unskip\endMakeFramed%
 \at@end@of@kframe}
\makeatother

\definecolor{shadecolor}{rgb}{.97, .97, .97}
\definecolor{messagecolor}{rgb}{0, 0, 0}
\definecolor{warningcolor}{rgb}{1, 0, 1}
\definecolor{errorcolor}{rgb}{1, 0, 0}
\newenvironment{knitrout}{}{} % an empty environment to be redefined in TeX

\usepackage{alltt}

\usepackage{amsmath}
\usepackage{hyperref}

\newcommand{\funname}[1]{\texttt{#1}}

\title{Noise in the sort matrix}
\author{Josh Burdick}
\IfFileExists{upquote.sty}{\usepackage{upquote}}{}
\begin{document}

\maketitle




\section{Some test data}

Before doing anything else, I define a tiny test data set, for these reasons:

\begin{itemize}

\item Any parameter tuning won't invalidate results on the real data. (Previously,
I was subsetting to, e.g., the most highly expressed genes.)

\item It may be easier to see if unmixing is working on a toy example.

\item Speed is not the biggest concern.

\end{itemize}

Here, the sort matrix is A, and the expression is X.
Note that since rows of A and X all add up to one, this is also
the case for the rows of B (they are all ``row-stochastic.'')


\begin{knitrout}
\definecolor{shadecolor}{rgb}{0.969, 0.969, 0.969}\color{fgcolor}\begin{kframe}
\begin{alltt}
\hlstd{A} \hlkwb{=} \hlkwd{matrix}\hlstd{(}\hlkwd{rgamma}\hlstd{(}\hlnum{40}\hlstd{,} \hlkwc{shape}\hlstd{=}\hlnum{1}\hlstd{,} \hlkwc{rate}\hlstd{=}\hlnum{1}\hlstd{),} \hlkwc{nrow}\hlstd{=}\hlnum{4}\hlstd{)}
\hlstd{A} \hlkwb{=} \hlstd{A} \hlopt{/} \hlkwd{apply}\hlstd{(A,} \hlnum{1}\hlstd{, sum)}
\hlkwd{round}\hlstd{(A,} \hlnum{2}\hlstd{)[,}\hlnum{1}\hlopt{:}\hlnum{5}\hlstd{]}
\end{alltt}
\begin{verbatim}
##      [,1] [,2] [,3] [,4] [,5]
## [1,] 0.20 0.21 0.01 0.11 0.07
## [2,] 0.09 0.05 0.26 0.06 0.04
## [3,] 0.08 0.16 0.26 0.05 0.07
## [4,] 0.08 0.13 0.03 0.05 0.19
\end{verbatim}
\begin{alltt}
\hlstd{X} \hlkwb{=} \hlkwd{matrix}\hlstd{(}\hlkwd{rgamma}\hlstd{(}\hlnum{200}\hlstd{,} \hlkwc{shape}\hlstd{=}\hlnum{1}\hlstd{,} \hlkwc{rate}\hlstd{=}\hlnum{1}\hlstd{),} \hlkwc{nrow}\hlstd{=}\hlnum{10}\hlstd{)}
\hlstd{X} \hlkwb{=} \hlstd{X} \hlopt{/} \hlkwd{apply}\hlstd{(X,} \hlnum{1}\hlstd{, sum)}
\hlstd{B} \hlkwb{=} \hlstd{A} \hlopt \hlstd{X}
\hlkwd{apply}\hlstd{(A,} \hlnum{1}\hlstd{, sum)}
\end{alltt}
\begin{verbatim}
## [1] 1 1 1 1
\end{verbatim}
\begin{alltt}
\hlkwd{apply}\hlstd{(X,} \hlnum{1}\hlstd{, sum)}
\end{alltt}
\begin{verbatim}
##  [1] 1 1 1 1 1 1 1 1 1 1
\end{verbatim}
\begin{alltt}
\hlkwd{apply}\hlstd{(B,} \hlnum{1}\hlstd{, sum)}
\end{alltt}
\begin{verbatim}
## [1] 1 1 1 1
\end{verbatim}
\end{kframe}
\end{knitrout}

One definition of ``working'' would be: if we unmix using a perturbed prior for
A (but using the {\em actual} sort matrix in simulating B), then the posterior
estimate for A is closer to the actual A than the prior.


\section{EP, with an approximate sort matrix}

So far, we've been treating the sort matrix $A$ as exactly known,
which seems like a weak assumption. Therefore, we assume that
its entries are drawn from a gamma distribution, and that
we have a constraint as before.

\begin{align*}
A_{i,j} &\sim \mathrm{Gamma}(\alpha_{i,j},\beta_{i,j}) \\
AX &= b
\end{align*}

If $A$ is known exactly, then I think I know how to approximate
the mean and variance of $X$. However, I don't know how to
modify that method if $A$ is approximate. Therefore, I'm
trying to use EP again.

\subsection{Breaking apart the constraint}

As a toy example of this, we can sample from the
marginals with two constraints, compared to applying
one constraint at a time. This is sort of like assuming
the covariance is diagonal.





If the system is fairly underdetermined, this seems like a
reasonable approximation. Presumably, it will do worse
in a more determined situation (that is, if the number
of sort fractions approaches the number of cells.) We
ignore this for now.

\subsection{Approximating one constraint}

Having done this, we estimate the marginals for each
constraint separately. We do this essentially by
scaling the unit simplex (as I described earlier.)
Without loss of generality, we can assume the constraint
adds up to 1:

\[
A_1 X_1 + A_2 X_2 + ... + A_n X_n = 1
\]

One wrinkle is that $A$ is not known exactly.

\begin{align*}
A_i X_i &\sim \mathrm{Beta}(1, n-1) \\
X_i &\sim {\mathrm{Beta}(1, n-1)}(\frac{1}{A_i})
\end{align*}


Since $A_i$ has a gamma distribution, $\frac{1}{A_i}$ has
an inverse gamma distribution. If $A$'s shape is at least
1, then we can find the mean and variance of $\frac{1}{A_i}$.
We then moment-match, using the fact that $E[XY] = E[X]E[Y]$.


\subsection{Applying EP}

We now (maybe) have a situation similar to that
described by Minka: several approximations which we can
iteratively improve.





\subsection{Comparison with sampling}


I don't have a strategy for sampling when there's noise
in the sort matrix.


\end{document}

