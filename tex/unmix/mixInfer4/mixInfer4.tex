\documentclass{article}\usepackage[]{graphicx}\usepackage[]{color}
%% maxwidth is the original width if it is less than linewidth
%% otherwise use linewidth (to make sure the graphics do not exceed the margin)
\makeatletter
\def\maxwidth{ %
  \ifdim\Gin@nat@width>\linewidth
    \linewidth
  \else
    \Gin@nat@width
  \fi
}
\makeatother

\definecolor{fgcolor}{rgb}{0.345, 0.345, 0.345}
\newcommand{\hlnum}[1]{\textcolor[rgb]{0.686,0.059,0.569}{#1}}%
\newcommand{\hlstr}[1]{\textcolor[rgb]{0.192,0.494,0.8}{#1}}%
\newcommand{\hlcom}[1]{\textcolor[rgb]{0.678,0.584,0.686}{\textit{#1}}}%
\newcommand{\hlopt}[1]{\textcolor[rgb]{0,0,0}{#1}}%
\newcommand{\hlstd}[1]{\textcolor[rgb]{0.345,0.345,0.345}{#1}}%
\newcommand{\hlkwa}[1]{\textcolor[rgb]{0.161,0.373,0.58}{\textbf{#1}}}%
\newcommand{\hlkwb}[1]{\textcolor[rgb]{0.69,0.353,0.396}{#1}}%
\newcommand{\hlkwc}[1]{\textcolor[rgb]{0.333,0.667,0.333}{#1}}%
\newcommand{\hlkwd}[1]{\textcolor[rgb]{0.737,0.353,0.396}{\textbf{#1}}}%

\usepackage{framed}
\makeatletter
\newenvironment{kframe}{%
 \def\at@end@of@kframe{}%
 \ifinner\ifhmode%
  \def\at@end@of@kframe{\end{minipage}}%
  \begin{minipage}{\columnwidth}%
 \fi\fi%
 \def\FrameCommand##1{\hskip\@totalleftmargin \hskip-\fboxsep
 \colorbox{shadecolor}{##1}\hskip-\fboxsep
     % There is no \\@totalrightmargin, so:
     \hskip-\linewidth \hskip-\@totalleftmargin \hskip\columnwidth}%
 \MakeFramed {\advance\hsize-\width
   \@totalleftmargin\z@ \linewidth\hsize
   \@setminipage}}%
 {\par\unskip\endMakeFramed%
 \at@end@of@kframe}
\makeatother

\definecolor{shadecolor}{rgb}{.97, .97, .97}
\definecolor{messagecolor}{rgb}{0, 0, 0}
\definecolor{warningcolor}{rgb}{1, 0, 1}
\definecolor{errorcolor}{rgb}{1, 0, 0}
\newenvironment{knitrout}{}{} % an empty environment to be redefined in TeX

\usepackage{alltt}

\usepackage{amsmath}
\usepackage{hyperref}

\newcommand{\funname}[1]{\texttt{#1}}

\title{Noise in the sort matrix}
\author{Josh Burdick}
\IfFileExists{upquote.sty}{\usepackage{upquote}}{}
\begin{document}

\maketitle



So far, we've been treating the sort matrix $A$ as exactly known,
which seems like a weak assumption. Therefore, we assume that
its entries are drawn from a gamma distribution, and that
we have a constraint as before.

\begin{align*}
A_{i,j} &\sim \mathrm{Gamma}(\alpha_{i,j},\beta_{i,j}) \\
AX &= b
\end{align*}

If $A$ is known exactly, then I think I know how to approximate
the mean and variance of $X$. However, I don't know how to
modify that method if $A$ is approximate. Therefore, I'm
trying to use EP again.

\subsection{Breaking apart the constraint}

As a toy example of this, we can sample from the
marginals with two constraints, compared to applying
one constraint at a time. This is sort of like assuming
the covariance is diagonal.





If the system is fairly underdetermined, this seems like a
reasonable approximation. Presumably, it will do worse
in a more determined situation (that is, if the number
of sort fractions approaches the number of cells.) We
ignore this for now.

\subsection{Approximating one constraint}

Having done this, we estimate the marginals for each
constraint separately. We do this essentially by
scaling the unit simplex (as I described earlier.)
Without loss of generality, we can assume the constraint
adds up to 1:

\[
A_1 X_1 + A_2 X_2 + ... + A_n X_n = 1
\]

One wrinkle is that $A$ is not known exactly.

\begin{align*}
A_i X_i &\sim \mathrm{Beta}(1, n-1) \\
X_i &\sim {\mathrm{Beta}(1, n-1)}(\frac{1}{A_i})
\end{align*}


Since $A_i$ has a gamma distribution, $\frac{1}{A_i}$ has
an inverse gamma distribution. If $A$'s shape is at least
1, then we can find the mean and variance of $\frac{1}{A_i}$.
We then moment-match, using the fact that $E[XY] = E[X]E[Y]$.



\subsection{Applying EP}

We now (maybe) have a situation similar to that
described by Minka: 







\subsection{Comparison with sampling}




I don't have a strategy for sampling when there's noise
in the sort matrix.



\section{Another way of estimating marginals}


To test this approach, we compare its marginals to those from
sampling (at least on a toy problem.)

First, we design a {\em very} toy problem:

\begin{knitrout}
\definecolor{shadecolor}{rgb}{0.969, 0.969, 0.969}\color{fgcolor}\begin{kframe}
\begin{alltt}
\hlstd{A} \hlkwb{=} \hlkwd{matrix}\hlstd{(}\hlkwd{sample}\hlstd{(}\hlnum{0}\hlopt{:}\hlnum{5}\hlstd{,} \hlnum{32}\hlstd{,} \hlkwc{replace}\hlstd{=}\hlnum{TRUE}\hlstd{),} \hlkwc{nrow}\hlstd{=}\hlnum{4}\hlstd{)}
\hlstd{A}
\end{alltt}
\begin{verbatim}
##      [,1] [,2] [,3] [,4] [,5] [,6] [,7] [,8]
## [1,]    2    3    1    2    5    0    2    4
## [2,]    1    4    4    3    4    4    3    5
## [3,]    4    5    0    2    3    4    5    0
## [4,]    0    1    0    4    5    1    4    1
\end{verbatim}
\begin{alltt}
\hlstd{x} \hlkwb{=} \hlkwd{sample}\hlstd{(}\hlkwd{c}\hlstd{(}\hlnum{0}\hlstd{,}\hlnum{1}\hlstd{,}\hlnum{2}\hlstd{,}\hlnum{4}\hlstd{,}\hlnum{8}\hlstd{),} \hlnum{8}\hlstd{,} \hlkwc{replace}\hlstd{=}\hlnum{TRUE}\hlstd{)}
\hlstd{x}
\end{alltt}
\begin{verbatim}
## [1] 4 4 8 1 4 0 8 2
\end{verbatim}
\begin{alltt}
\hlstd{b} \hlkwb{=} \hlkwd{as.vector}\hlstd{(A} \hlopt \hlstd{x)}
\hlstd{b}
\end{alltt}
\begin{verbatim}
## [1]  74 105  90  62
\end{verbatim}
\end{kframe}
\end{knitrout}



\begin{knitrout}
\definecolor{shadecolor}{rgb}{0.969, 0.969, 0.969}\color{fgcolor}\begin{kframe}


{\ttfamily\noindent\bfseries\color{errorcolor}{\#\# Error in eval(expr, envir, enclos): object 'x.infer' not found}}

{\ttfamily\noindent\bfseries\color{errorcolor}{\#\# Error in cdaCppCore(A, b, x0, Z, num.samples, thinning): object 'x0' not found}}\end{kframe}
\end{knitrout}



We compute marginals using this method.

\begin{knitrout}
\definecolor{shadecolor}{rgb}{0.969, 0.969, 0.969}\color{fgcolor}\begin{kframe}
\begin{alltt}
\hlstd{x.infer} \hlkwb{=} \hlkwd{approx.region.simplex.transform}\hlstd{(A,b)}
\end{alltt}


{\ttfamily\noindent\bfseries\color{errorcolor}{\#\# Error in eval(expr, envir, enclos): could not find function "{}approx.region.simplex.transform"{}}}\begin{alltt}
\hlkwd{round}\hlstd{(x.infer,}\hlnum{2}\hlstd{)}
\end{alltt}


{\ttfamily\noindent\bfseries\color{errorcolor}{\#\# Error in eval(expr, envir, enclos): object 'x.infer' not found}}\begin{alltt}
\hlkwd{round}\hlstd{(}\hlkwd{as.vector}\hlstd{(A} \hlopt \hlstd{x.infer[}\hlstr{"m"}\hlstd{,]),} \hlnum{2}\hlstd{)}
\end{alltt}


{\ttfamily\noindent\bfseries\color{errorcolor}{\#\# Error in as.vector(A \%*\% x.infer["{}m"{}, ]): object 'x.infer' not found}}\end{kframe}
\end{knitrout}

So, at least the mean matches the constraints.

\begin{knitrout}
\definecolor{shadecolor}{rgb}{0.969, 0.969, 0.969}\color{fgcolor}\begin{kframe}
\begin{alltt}
\hlstd{x} \hlkwb{=} \hlkwd{rgamma}\hlstd{(}\hlnum{1e5}\hlstd{,} \hlkwc{shape}\hlstd{=}\hlnum{1}\hlstd{,} \hlkwc{scale}\hlstd{=}\hlnum{5}\hlstd{)}
\hlstd{y} \hlkwb{=} \hlkwd{rgamma}\hlstd{(}\hlnum{1e5}\hlstd{,} \hlkwc{shape}\hlstd{=}\hlnum{2}\hlstd{,} \hlkwc{scale}\hlstd{=}\hlnum{3}\hlstd{)}
\hlkwd{par}\hlstd{(}\hlkwc{mfrow}\hlstd{=}\hlkwd{c}\hlstd{(}\hlnum{1}\hlstd{,}\hlnum{3}\hlstd{))}
\hlkwd{hist}\hlstd{(x)}
\hlkwd{hist}\hlstd{(y)}
\hlkwd{hist}\hlstd{(x}\hlopt{*}\hlstd{y)}
\end{alltt}
\end{kframe}
\includegraphics[width=\maxwidth]{figure/sample2-1} 

\end{knitrout}

\begin{knitrout}
\definecolor{shadecolor}{rgb}{0.969, 0.969, 0.969}\color{fgcolor}
\includegraphics[width=\maxwidth]{figure/samplePlot1-1} 

\end{knitrout}


\end{document}

