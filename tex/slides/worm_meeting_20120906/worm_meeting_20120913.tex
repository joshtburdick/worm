\documentclass[serif,9pt]{beamer}
\usepackage[latin1]{inputenc}
\usepackage{multirow}
\renewcommand{\sfdefault}{times}

% \usetheme{Warsaw}

\graphicspath{
{/home/jburdick/gcb/work/svnroot/trunk/tex/image/}
{/home/jburdick/gcb/work/svnroot/trunk/tex/thesis/proposal/image/}
{/home/jburdick/gcb/work/svnroot/trunk/src/}
{/home/jburdick/gcb/}
}
\beamertemplatenavigationsymbolsempty

\title{Where are genes expressed in the {\em C. elegans} embryo?}
\author{Josh Burdick \\ Murray Lab}
\date{September 13, 2012}
\begin{document}

\begin{frame}
\titlepage
\end{frame}

\begin{frame}{Where a gene is expressed is important}

\begin{centering}
\includegraphics[width=\textwidth]{notch_example_1.png}
\end{centering}

\hfill Neves and Priess, 2005

\end{frame}

\begin{frame}{Where {\em many} genes are expressed is important}

\includegraphics[width=\textwidth]{yanai2008.png}

\hfill Yanai, 2008

\end{frame}

\begin{frame}

\includegraphics[width=\textwidth]{image/movie/P0_1200x400/20110731_RW10434_L1.png}

\end{frame}

\begin{frame}{Where and when is each gene expressed in development?}

\begin{itemize}

\item What genes are enriched in cells that express some TF?

\pause 

\item What genes are enriched in cells expressing several of the same TFs?

\pause

\item What does this imply about how genes are regulated?

\end{itemize}

\end{frame}

\begin{frame}{Samples sorted}

FIXME: include Travis' FACSlog

GFP-protein fusions

RFP-promotor fusions

Controls

\end{frame}

\begin{frame}{How reproducible are the data?}

\includegraphics[width=0.8\textwidth]{git/unmix/seq/quant/replicateScatterPlot.png}

\end{frame}

\begin{frame}{Which samples are similar to each other?}

\includegraphics[width=1\textwidth]{git/unmix/seq/cluster/pvclust.pdf}

\end{frame}

\begin{frame}{Are some cells depleted from the sorting process?}

\includegraphics[width=0.8\textwidth]{git/unmix/seq/FACS/ungated_and_singlets.pdf}

\end{frame}


\begin{frame}{Some genes are mostly expressed at a particular time}

\includegraphics[height=0.8\textheight]{git/unmix/seq/timing/timeMeanAndSD.pdf}

\hfill Boeck and Waterston {\em et al}, 2012

\end{frame}

\begin{frame}{Picking time-specific genes}

\includegraphics[width=\textwidth]{time_figure.png}

\hfill Boeck and Waterston {\em et al}, 2012

\end{frame}

\begin{frame}{Which embryonic times are included in the sorting?}

\includegraphics[width=0.8\textwidth]{git/unmix/seq/timing/perStageMarkersUngated.pdf}

\end{frame}


\begin{frame}{{\em cnd-1}}
\begin{minipage}{0.4\textwidth}
\includegraphics[width=\textwidth]{R/unmix/sort_paper/seq/fraction/plots_and_stats/cnd-1.pdf}
\end{minipage}
\begin{minipage}{0.58\textwidth}
\begin{table}[!tbp]\scriptsize
\begin{tabular}{lll}
anatomy term & \# genes & adjusted $p$ \\
\hline
germ line & 295 & $10^{-17}$ \\
QR.a/p & 8 & 0.005 \\
cc(A/P)(L/R) & 8 & 0.005 \\
D & 25 & 0.0001 \\
C(a/p)p & 25 & 0.0002 
\end{tabular}
\end{table}
\end{minipage}

\includegraphics[width=\textwidth]{image/movie/P0_1200x400/20110731_RW10434_L1.png}

\end{frame}

\begin{frame}{{\em unc-130}}

\begin{minipage}{0.4\textwidth}
\includegraphics[width=\textwidth]{R/unmix/sort_paper/seq/fraction/plots_and_stats/unc-130.pdf}
\end{minipage}
\begin{minipage}{0.58\textwidth}
\begin{table}[!tbp]\scriptsize
\begin{tabular}{lll}
anatomy term & \# genes & adjusted $p$ \\
\hline

\end{tabular}
\end{table}
\end{minipage}

\includegraphics[width=\textwidth]{image/movie/P0_1200x400/20110927_RW11144_L4.png}

\end{frame}

\begin{frame}{{\em ceh-26}}

\begin{minipage}{0.4\textwidth}
\includegraphics[width=\textwidth]{R/unmix/sort_paper/seq/fraction/plots_and_stats/ceh-26.pdf}
\end{minipage}
\begin{minipage}{0.58\textwidth}
\begin{table}[!tbp]\scriptsize
\begin{tabular}{lll}
anatomy term & \# genes & adjusted $p$ \\
\hline
excretory secretory system & 5 & 0.006 
\end{tabular}
\end{table}
\end{minipage}

\includegraphics[width=\textwidth]{image/movie/P0_1200x400/20120401_JIM122_L3.png}

\end{frame}

\begin{frame}{{\em ceh-6}}

\begin{minipage}{0.4\textwidth}
\includegraphics[width=\textwidth]{R/unmix/sort_paper/seq/fraction/plots_and_stats/ceh-6.pdf}
\end{minipage}
\begin{minipage}{0.58\textwidth}
\begin{table}[!tbp]\footnotesize
\begin{tabular}{lr}
\multicolumn{1}{l}{gene}&\multicolumn{1}{c}{enrichment}\tabularnewline
\hline
axl-1&$37.1$\tabularnewline
oac-49&$17.0$\tabularnewline
nlp-11&$14.3$\tabularnewline
fip-2&$11.9$\tabularnewline
tpst-2&$11.8$\tabularnewline
srh-67&$ 9.9$\tabularnewline
nas-3&$ 9.0$\tabularnewline
fbxb-95&$ 8.5$\tabularnewline
oac-8&$ 8.1$\tabularnewline
hlh-34&$ 8.1$\tabularnewline
\hline
\end{tabular}
\end{table}

\end{minipage}

\begin{table}\footnotesize
\begin{tabular}{llr}
GO category & genes & $p$ value \\
\hline
G-protein coupled receptor signalling & {\em srw-113, gnrr-7, srg-31} (12 total) & $10^{-4}$ \\
axon guidance & {\em ceh-17, zag-1, unc-6} & $10^{-3}$ \\
\end{tabular}
\end{table}

FIXME add tree here...
% \includegraphics[width=0.15\textwidth,angle=270]{/gpfs/fs0/l/murr/trees/20100730_ceh-6_L1.png}
\end{frame}

\begin{frame}{Genes enriched in sorted fractions}

\begin{table}\footnotesize
\begin{tabular}{lllll}
sort & \% of & num. genes & enriched GO & enriched \\
marker & cells & enriched & categories & tissues \\
\hline

\end{tabular}
\end{table}

\end{frame}

\begin{frame}{Clusters of genes enriched in the same sort fractions}



\end{frame}

\begin{frame}{Fourteen reporters distinguish many cells}

\begin{figure}
\begin{centering}
\includegraphics[width=0.9\textwidth]{R/unmix/sort_paper/plot/distinguishableCells.pdf}

\end{centering}
\end{figure}

\end{frame}

\begin{frame}{Cell-type-specific expression data can suggest regulatory relationships}

\includegraphics[height=0.8\textheight]{imported/novoshtern2011fig3.jpg}

Human hematopoeisis (Novoshtern et al., 2011)

{\small flow sorted blood cells into groups, then profiled expression of each group}

\end{frame}










\end{document}

