\documentclass{article}
% \usepackage{amsfonts}
% \usepackage[cm]{fullpage}
\usepackage{graphicx}
\usepackage{multirow}
% \usepackage{hyperref}
\graphicspath{
{/home/jburdick/gcb/work/svnroot/trunk/src/}
{/home/jburdick/gcb/}
}
\begin{document}

\title{Expression patterns and their regulation}
\author{Josh Burdick}
\maketitle

\section*{Unmixing}

Previously, we described an approach for estimating expression in
groups of cells, from measurements in a subset of cells.

In order to control for missing cells, or errors in the sort matrix,
we only used sort fractions for which both positive and negative
sort fractions are available. Currently, we have twelve such fractions
available:
{\em ceh-26, ceh-27, ceh-36, ceh-6, cnd-1, F21D5.9, mir-57, mls-2, pal-1,
pha-4, ttx-3,} and {\em unc-130.}


We converted each gene's relative expression to a ``proportion of total
expression'' between 0 and 1.


Higher expression measurements tend to
be more reproducible (FIXME CITE). We see this in the samples for which
we have replicates.

Based on this, we modelled the measurement error, based on the read count,
so that the measurement standard deviation was
narrower for genes with higher measured
expression.


replicateNoise.png



It seems that ideally, we should be estimating the sort matrix, as well
as the expression matrix. It is not currently clear how to do that.
Also, incorporating the timeseries data would be helpful.


\section*{Clustering}






\section*{Motif analysis}

We searched for motifs which might regulate expression.


\subsection*{Known binding motifs}

We used a subset of the motifs distributed with the MEME software.
This includes a set of FIXME motifs for human and mouse transcription
factors obtained recently using
high-throughput SELEX experiments.




\subsection*{{\em De novo} motif search}

We searched for novel motifs using three different methods:
BioProspector, which uses Gibbs sampling, and MEME, which uses EM.

We searched for motifs in the upstream intergenic regions.
We included
up to 5 kb upstream, and at least 500 bp.

We also searched the conserved portions of those regions,
requiring a PhastCons score of 7-way conservation with other nematodes
of at least 0.5.

One measure of how well



\subsection*{Motif clustering}

Many different transcription factors recognize essentially the same
motif. Also, many of the motifs found {\em de novo} will be duplicates,
because a transcription factor is active in multiple clusters, or because,
for instance, motif finders based on Gibbs sampling may sample the same
motif multiple times. Therefore, we need to reduce the motifs to a
non-redundant set.

We used the MotIV R package to compute similarities between known
motifs, using Pearson correlation coefficient as a similarity score.
Here is an example of such a cluster. In this case, a motif found
{\em de novo} is very similar to the known mouse {\em Ascl2} motif.


Ascl2cluster.pdf



We then grouped together any pair of motifs having Pearson correlation
$ > FIXME$, and kept one of the motifs arbitrarily.

\section*{Enrichment of motifs}

If a particular transcription factor is important in regulating
a particular cluster of genes, we might expect to see a motif
enriched upstream of the genes in that cluster.







\begin{center}
\begin{tabular}{lr|rr}
\hline
                 & \# & \# of clusters with an enriched & \# of clusters with an enriched \\
Clustering method & clusters & known motifs & {\em de novo} motifs \\
\hline
Hierarchical & 50 & 9 & 9 \\
not including time & 100 & 7 & 8 \\
            & 200 & 7 & 9 \\
\hline
Hierarchical & 50 & 8 & 12 \\
including time & 100 & 9 & 11 \\
  & 200 & 11 & 12 \\
\hline
WGCNA not & n & 9 & 11 \\
including time & n & 8 & 9 \\
\hline
WGCNA & n & 10 & 9 \\
including time & n & 8 & 12 \\
\hline
\end{tabular}
\end{center}

\section*{Summary}





\section*{Future work}

We have predicted where many genes are expressed.





\end{document}

