\documentclass{article}
\usepackage{amsfonts}
\usepackage{amsmath}
\usepackage[cm]{fullpage}

\usepackage{pdfpages}

\usepackage{hyperref}

\usepackage{helvet}
\renewcommand{\familydefault}{\sfdefault}

\usepackage[margin=10pt,justification=justified]{caption}


\usepackage{graphicx}
\usepackage[space]{grffile}
\graphicspath{
{/home/jburdick/gcb/}
}

\renewcommand{\thefigure}{S\arabic{figure}} 

\begin{document}
%% \pagenumbering{gobble}

\begin{figure}

\caption{Expression of fourteen reporters used for FACS sorting,
measured by lineage tracing.}
\end{figure}

\clearpage
\includepdf[pages={1-},scale=1,offset=-2in 0in,pagecommand={\thispagestyle{plain}}]{writing/sort paper/trees/S1 Sort markers.pdf}
\clearpage

\begin{figure}
\includegraphics[width=\textwidth]
{git/sort_paper/plot/cnd1Scatterplot.pdf}
\caption{
Enrichments for two replicates of {\em cnd-1} sorting.}
\end{figure}
\clearpage


\begin{figure}
\includegraphics[width=\textwidth]
{git/sort_paper/FACS/matchedVsSingletControl.png}
\caption{
Comparison of enrichments using a matched control.
$x$-axis: enrichment of (+) sample, compared to the corresponding
(-) sample. $y$-axis:
enrichment of (+) sample compared to singlet control.
rather than the non-expressing (-) sample corresponding to a given expressing (+) experiment. ({\em hlh-16} and {\em irx-1} are omitted, as they
lacked a matching (-) control.)}
\end{figure}
\clearpage

\begin{figure}
\includegraphics[width=\textwidth]
{git/sort_paper/FACS/timing/geneExprMeanAndSD.pdf}
\caption{
Mean and standard deviation of when genes were expressed,
using expression timeseries from
(The modENCODE Consortium 2010).
Genes below the horizontal line were considered ``time-specific'', and
used in plotting enrichments relative to time.
}
\end{figure}
\clearpage

\begin{figure}
\caption{Enrichments for selected pairs of samples, calculated for
time-specific genes from (The modENCODE Consortium 2010).} 
\end{figure}
\clearpage

\includepdf[pages={1-5},scale=1,pagecommand={\thispagestyle{plain}}]{git/sort_paper/FACS/timing/exprByTime.pdf}

\clearpage

\begin{figure}
\includegraphics[width=\textwidth]
{git/sort_paper/FACS/regress/facsRegress.pdf}
\caption{
Accuracy predicting the expression in ungated samples 
using only (+) and (-) samples ($x$ axis), or using the (+), (-), and
singlet control samples ($y$ axis).}
\end{figure}
\clearpage

\begin{figure}
\includegraphics[width=0.95\textwidth]
{git/sort_paper/unmix/pseudoinverseEnrichment/crossvalidationAccuracy.png}
\caption{
Unmixing cross-validation accuracy. For each sort marker $s$, the $x$ axis
shows measured enrichment computed from the $s$ (+) and $s$ (-) samples.
The $y$ axis shows the enrichment predicted for $s$, based on the measured
expression of all samples except $s$.
}
\end{figure}
\clearpage

\begin{figure}
\includegraphics[width=\textwidth]
{git/sort_paper/cluster/comparison/exprAndClustering.pdf}
\caption{
Correlation of expression patterns for genes in different clusters and the same cluster, for (A) 119 embryonic expression patterns from (Murray et al. 2012) and (B) 82 expression patterns from L1 stage larvae (Liu et al. 2009).
}
\end{figure}


\begin{figure}
\includegraphics[width=\textwidth]
{git/sort_paper/cluster/annotation/tissueSpecificitySupplemental.pdf}
\caption{Mean expression and mean absolute enrichments of clusters
(as in Figure 4E), with all clusters labelled.
}
\end{figure}
\clearpage


\begin{figure}
\includegraphics[width=\textwidth]
{git/sort_paper/plot/enrichment/stackedPlots/hier.300.pdf}
\caption{Annotation enrichment as in Figure 5, but for all clusters.
}
\end{figure}
\clearpage

\begin{figure}
\includegraphics[height=0.9\textheight]
{git/sort_paper/enrichment/cutoffOptimize.pdf}
\caption{Reproducibility of enrichments, at different cutoffs. For
each cutoff on the $x$ axis in one sample, the $y$ axis shows the
fraction of genes which were enriched in a replicate experiment.
}
\end{figure}
\clearpage

\end{document}

